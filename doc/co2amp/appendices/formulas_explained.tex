\chapter{Selected formulas explained}
\label{appendix:formulas_explained}

\subsection*{Equation \ref{eq:z_jk}}

Eq. \ref{eq:z_jk} defines the fraction $z_{jk}$ of discharge energy spent in inelastic collisions:

\begin{equation*}
z_{jk} = 10^{16} \frac{y_j u_{jk} \omega _{jk}} {\left( \frac{\xi\mathcal{E}}{\mathcal{N}} \right) v_d}  
\end{equation*}
where $y_j[\text{-}]$ is the relative concentration of a component in the gas mixture, $u_{jk}[\text{eV}]$ is the transferred energy per electron-molecule collision, collision rate constant $\omega _{jk}[\text{cm}^3\cdot\text{s}^{-1}]$ divided by electron drift speed $v_d[\text{cm}\cdot\text{s}^{-1}]$ is the collision cross-section ($[\text{cm}^2]$), $\mathcal{E}[10^{-16}\text{V}\cdot\text{cm}^{-1}]$ is the electric field, $\xi[\text{eV}\cdot\text{V}^{-1}]$ is the energy gained by electron moved across an electric potential difference of 1~V,  and $\mathcal{N}[\text{cm}^{-3}]$ is the total absolute concentration of the gas mixture.

The physical meaning of $\xi\mathcal{E}$ is the energy (in eV) gained by an electron after passing 1~cm in the electric field $\mathcal{E}$. By definition of electronvolt, $\xi = 1$ and is thus omitted in Eq. \ref{eq:z_jk}.


\subsection*{Pumping rate constants in equations \ref{eq:dedt} and \ref{eq:dedt_rates}}

Pumping rate constant is the number of quanta added to a given vibrational mode per unit of time per molecule.

\begin{equation*}
p_e = \frac{1}{E_v[\text{J}]} \times \frac{1}{N[\text{cm}^{-3}] n[\text{-}] y[\text{-}]} \times q[\text{-}] W[\text{J}\cdot \text{s}^{-1} \cdot \text{cm}^{-3}]
\end{equation*}
where $E_v$ is the energy of the vibrational quanta: 4.665e-20~J (2349~{cm$^{-1}$}) for $\nu_3$ mode of \ce{CO2} (and roughly same for \ce{N2} vibration), and  1.325e-20 J (667 {cm$^{-1}$}) for $\nu_2$ mode; $N$=2.7e19 {cm$^{-3}$} is the density of gas molecules under normal conditions (1~bar, 273~K); $q$ is the fraction of discharge energy deposited in the corresponding vibration; $n$ is the correction factor for molecular density at the conditions different from 'normal'; $y$ is the relative concentration of the gas in the mixture; $W$ is the discharge power density.

Combining the constants and switching to kW/cm$^3$ for power density and \si{\micro\second^{-1}} for the rate constants we get the formulas given in the model description:

\begin{equation*}
p_{e4} = 0.8\times 10^{-3} \frac{q_4}{n y_2} W(t);\quad p_{e3} = 0.8\times 10^{-3}\frac{q_3}{n y_1} W(t);\quad p_{e2} = 2.8\times 10^{-3}\frac{q_2}{n y_1} W(t);
\end{equation*}
